\documentclass[a4paper]{article}
\usepackage{tombstone}
\usepackage{float}
\usepackage{listings}
\usepackage{xcolor}

\definecolor{codegray}{gray}{0.95}
\lstset{ %
    basicstyle=\footnotesize,        % the size of the fonts that are used for the code
    breakatwhitespace=false,         % sets if automatic breaks should only happen at whitespace
    backgroundcolor=\color{codegray}, xleftmargin=0.7cm,
    breaklines=true,                 % sets automatic line breaking
    captionpos=b,                    % sets the caption-position to bottom
    escapeinside={\%*}{*)},          % if you want to add LaTeX within your code
    extendedchars=true,              % lets you use non-ASCII characters; for 8-bits encodings only, does not work with UTF-8
    frame=tlbr, framesep=0.2cm, framerule=0pt,
                     % adds a frame around the code
    keepspaces=true,                 % keeps spaces in text, useful for keeping indentation of code (possibly needs columns=flexible)
    keywordstyle=\bf,       % keyword style
    %numbers=left,                    % where to put the line-numbers; possible values are (none, left, right)
    %numbersep=5pt,                   % how far the line-numbers are from the code
    %numberstyle=\tiny, % the style that is used for the line-numbers
    showspaces=false,                % show spaces everywhere adding particular underscores; it overrides 'showstringspaces'
    showstringspaces=false,          % underline spaces within strings only
    showtabs=false,                  % show tabs within strings adding particular underscores
    tabsize=4,                     % sets default tabsize to 2 spaces
    basicstyle=\linespread{1.1}\ttfamily\footnotesize,
}

\title{Tombstone package\\{\normalsize v. 0.1}}
\author{Mathias Jakobsen}
\date{\today}

\begin{document}
    \maketitle

    \noindent The tombstone package is a \LaTeX\, package for drawing tombstone diagrams (or T-diagrams).
    
    \tableofcontents

    \section{Introduction}
    Figure \ref{fig:shapes} shows the four shapes that are available for use in tombstone diagrams. In order, the four different shapes represent programs, compilers, interpreters and machines. These shapes can be combined into `puzzles' that represent the transformation of a program written in a source language into the same program written in a target language.

    \begin{figure}[h]
        \centering
        \begin{tikzpicture}
            \begin{tombstonediagram}
                \program{prg}{P}{C}{}
                \compiler[right=0pt of prg]{cmp}{C}{M}{M}{}
                \interpreter[right=0pt of cmp]{int}{CLR}{M}{}
                \machine[right=0pt of int]{mac}{M}{}
                %\draw [fill=red] (prog-1-1.north west) circle (1mm);
            \end{tombstonediagram}
        \end{tikzpicture}
        \caption{Shows the possible shapes for use in a tombstone diagram.}
        \label{fig:shapes}
    \end{figure}

    \subsection{Installation}
    Installation of this package is very simple. Simply move the \lstinline|tombstone.sty| file to the root folder of your project, and include the package using the command \lstinline|\usepackage{tombstone}|.

    \section{Creating diagrams}
    This section will describe the commands necessary to create diagrams.

    \subsection{Shapes}
    The four shapes are created using matrices, which defines points that can be used to attach different shapes to each other. The matrix used to create the program shape can be seen below on figure \ref{fig:matrix}. The red disk on the figure shows the default anchor of the shape. The location of these anchors differs from shape to shape, according to how they are commonly placed in a diagram. The default anchor location can be overwritten by specifying a new anchor as a named argument to the shape.

    Each corner in the matrix can be used as a anchor point by following the convention \lstinline|<shapeName>-<row>-<column>[.north/south west/east]|. For example, to draw the red disc in the figure, the point \lstinline|prg-2-2.north west| is used to denote the north western corner of the middle column of the bottom row.

    \begin{figure}[H]
        \begin{center}
            \begin{tikzpicture}
                \begin{tombstonediagram}
                    \program[drawmatrix]{prg}{P}{C}{}
                    \draw [fill=red] (prg-2-2.north west) circle (1mm);
                \end{tombstonediagram}
            \end{tikzpicture}
            \caption{}
            \label{fig:matrix}
        \end{center}
        \begin{lstlisting}[caption={Code for generating figure \ref{fig:matrix}}]
\begin{tikzpicture}
    \begin{tombstonediagram}
        \program[drawmatrix]{prg}{P}{C}{}
        \draw [fill=red] (prg-2-2.north west) circle (1mm);
    \end{tombstonediagram}
\end{tikzpicture}\end{lstlisting}
    \end{figure}

    \noindent The four different shapes can be created using the following commands. The placement specifier for the first shape in a diagram can be omitted, as the shape will be placed in the upper left corner by default. Otherwise the placement specifier is used for choosing what other shape to place the shape unto.

\begin{lstlisting}[caption={Code for different shapes}]
\begin{tikzpicture}
    \begin{tombstonediagram}
        \program{<shape name>}{<program name>}{<language>}
            {<placement>}
        \compiler{<shape name>}{<source language>}
            {<target language>}{<placement>}
        \interpreter{<shape name>}{<language for interpreter>}
            {<placement>}
        \machine{<shape name>}{machine language}{<placement>}
    \end{tombstonediagram}
\end{tikzpicture}\end{lstlisting}

    \begin{figure}[htpb]
        \centering
        \begin{tikzpicture}
            \begin{tombstonediagram}
                \program[drawmatrix]{prg}{P}{C}{}
                \compiler[drawmatrix,right=0pt of prg]{cmp}{C}{M}{M}{}
                \interpreter[drawmatrix, right=0pt of cmp]{int}{CLR}{M}{}
                \machine[drawmatrix, right=0pt of int]{mac}{M}{}
                \draw [fill=red] (prg-2-2.north west) circle (1mm);
                \draw [fill=red] (cmp-1-1.north west) circle (1mm);
                \draw [fill=red] (int-1-1.north west) circle (1mm);
                \draw [fill=red] (mac-1-1.north west) circle (1mm);
            \end{tombstonediagram}
        \end{tikzpicture}
        \caption{Shows the matrices and anchors of the different possible shapes.}
        \label{fig:anchors}
    \end{figure}

    \subsection{Connecting shapes}
    Showing how shapes are connected is best done through examples. The following listing generates the diagram shown on figure \ref{fig:connections}.

\begin{lstlisting}[caption=Code for generating figure \ref{fig:connections}]
\begin{tikzpicture}
    \begin{tombstonediagram}
        \compiler{cmp}{Java}{JVM}{JVM}{}
        \program[anchor=prg-2-2.north east]{prg}{Tetris}{Java}
            {at (cmp-1-1.north west)}
        \interpreter{int}{JVM}{x86}{at (cmp-2-2.south west)}
        \program{prgo}{Tetris}{JVM}{at (cmp-1-3.north east)}
        \machine{mac}{x86}{at (int-2-1.south west)}
    \end{tombstonediagram}
\end{tikzpicture}\end{lstlisting}

    \begin{figure}[H]
        \centering
        \resizebox{.6\textwidth}{!}{%
        \begin{tikzpicture}
            \begin{tombstonediagram}
                \compiler{cmp}{Java}{JVM}{JVM}{}
                \program[anchor=prg-2-2.north east]{prg}{Tetris}{Java}{at (cmp-1-1.north west)}
                \interpreter{int}{JVM}{x86}{at (cmp-2-2.south west)}
                \program{prgo}{Tetris}{JVM}{at (cmp-1-3.north east)}
                \machine{mac}{x86}{at (int-2-1.south west)}
            \end{tombstonediagram}
        \end{tikzpicture}
        }
        \caption{Connected shapes}\label{fig:connections}
    \end{figure}

    \subsubsection{Highlighting shapes}
    When creating diagrams, it may be necessary to highlight shapes in the diagram. Using the commands \lstinline|\hightlight<shape type>{<shape name>}| it is possible to do so.

    Appending the following 4 lines to the former example, generates figure \ref{fig:highlights}.
\begin{lstlisting}
\highlightprogram{prg}
\highlightcompiler{cmp}
\highlightinterpreter{int}
\highlightmachine{mac}
\end{lstlisting}



    \begin{figure}[H]
        \centering
        \resizebox{.6\textwidth}{!}{%
        \begin{tikzpicture}
            \begin{tombstonediagram}
                \compiler{cmp}{Java}{JVM}{JVM}{}
                \program[anchor=prg-2-2.north east]{prg}{Tetris}{Java}{at (cmp-1-1.north west)}
                \interpreter{int}{JVM}{x86}{at (cmp-2-2.south west)}
                \program{prgo}{Tetris}{JVM}{at (cmp-1-3.north east)}
                \machine{mac}{x86}{at (int-2-1.south west)}
                \highlightmachine{mac}
                \highlightinterpreter{int}
                \highlightcompiler{cmp}
                \highlightprogram{prg}
            \end{tombstonediagram}
        \end{tikzpicture}
        }
        \caption{Connected shapes with highlighting}\label{fig:highlights}
    \end{figure}
\end{document}

